The equation of states of (\gls{EoS}) of the spin polarized, asymmetric nuclear matter (\gls{NM}) 
is studied within the nonrelativistic Hartree-Fock (\gls{HF}) formalism using realistic choices 
for the in-medium (density dependent) nucleon-nucleon (\gls{NN}) interaction, dubbed as CDM3Y4, CDM3Y5, 
CDM3Y6 and CDM3Y8. Two scenarios for the density dependence of the spin polarization $\Delta$ 
of baryons in NM are considered, and the obtained HF results are compared with the empirical 
constraints for the nuclear symmetry energy given by the nuclear structure studies and the 
astrophysical observations of the binary NS merger (GW170817 and GW190425). A partial spin 
polarization of baryons ($\Delta < 1.0$) at low baryon densities seems more reasonable, with 
the HF results for the symmetry energy and incompressibility of NM being quite close to the 
empirical values. The mean-field based EoS of asymmetric NM is used further to construct 
the \textbeta-stable neutron star (\gls{NS}) matter of strongly interacting baryons (protons 
and neutrons), electrons, and muons.

The EoS of NS matter over a wide range of baryon densities is used as input for the calculation 
of the macroscopic configuration of \gls{NS} within the framework of General Relativity (\gls{GR}), 
like the gravitational mass $M$, radius $R$, gravito-electric and gravito-magnetic tidal deformability. 
Given the empirical constrains inferred from the gravitational-wave signals of GW170817 and the mass 
limit of the heaviest pulsars observed, we conclude that the EoS of NS matter given by the CDM3Y6 
and CDM3Y8 versions of the in-medium NN interaction is the most appropriate for the study of NS. 
The Love numbers of the tidal deformation of \gls{NS} in a binary system are calculated up 
to the 4\textsuperscript{th} order and a correlation of the tidal deformability of NS with 
its gravitational mass is shown. \\ [5mm]
\textbf{Keywords:} Neutron star, Magnetar, Spin polarization, Equation of state, Nuclear matter, 
Tidal deformability, Love number, Gravitational wave.
