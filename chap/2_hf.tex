\section{Nucleon-Nucleon Interaction}%
\label{sec:nucleon_nucleon_interaction}

Due to the lack of a exact theory to describe the nucleon-nucleon (\gls{NN}) interaction, a model need to be imposed and fit with experimental measurement or theoretical calculation results. Plus, for a system as massive as a \gls{NS}, deducing the \gls{EoS} using the \emph{ab initio} method, i.e. solving the Schr\"{o}dinger equation over all particles, is simply impossible, therefore an \emph{effective interaction} must be used \cite{greiner1996nuclear}. In this section, we only limit ourselves to two-body interaction, thus, the \gls{NN} potential can be expressed in the form of
\begin{equation}
        v = v(\bm{r}, \bm{r'}, \bm{p}, \bm{p'}, \bm{\sigma}, \bm{\sigma'}, \bm{\tau}, \bm{\tau'})
        \label{eff_interaction}
\end{equation}
where the primed and unprimed variables indicate the properties of 2 nucleons respectively, in which $\bm{r}$ is the particle's position, $\bm{p}$ is its momentum, $\bm{\sigma}$ is its intrinsic spin and $\bm{\tau}$ is its isospin.\par
The functional form of $v$ in \eqref{eff_interaction} cannot freely take any form but is constrained by many invariance requirements \cite{greiner1996nuclear}
\begin{itemize}
        \item \textbf{Translational invariance:} The \gls{NN} potential should only depend on the \emph{relative position} of the two particle but not their explicit positions, thus we can reduce \eqref{eff_interaction} to
                \begin{equation}
                        v = v(\bm{r}-\bm{r'},\bm{p},\bm{p'},\bm{\sigma},\bm{\sigma'},\bm{\tau},\bm{\tau'}) = v(\bm{r},\bm{p},\bm{p'},\bm{\sigma},\bm{\sigma'},\bm{\tau},\bm{\tau'})
                \end{equation}
                with at the last expression, we redefine $\bm{r}$ as the relative position vector.

        \item \textbf{Galilei invariance:} The potential should also be invariant under transformation between inertial frame of reference, which requires that only the relative momentum $\bm{p}-\bm{p'}$ is depended, i.e.
                \begin{equation}
                        v = v(\bm{r},\bm{p},\bm{\sigma},\bm{\sigma'},\bm{\tau},\bm{\tau'})
                        \label{eq2-3}
                \end{equation}
                where here we denote $\bm{p}$ as the relative momentum.

        \item \textbf{Rotational invariance:} The potential should be constructed such that the total angular momentum is zero.
                
        \item \textbf{Isospin invariance:} The \gls{NN} interaction potential needs to be invariant under rotation in isospin space, in other words, it can only depend on the isospin-independent terms and the terms with $\bm{\tau}\cdot\bm{\tau'}$. Therefore, we can split the potential into
                \begin{equation}
                        v = v_0 (\bm{r},\bm{p},\bm{\sigma},\bm{\sigma'}) + v_1 (\bm{r},\bm{p},\bm{\sigma},\bm{\sigma'}) \bm{\hat{\tau}}\cdot\bm{\hat{\tau}'}
                \end{equation}

        \item \textbf{Parity invariance:} The \gls{NN} interaction potential is also expected to be invariant under the action of parity operator, i.e. changing the sign of spatial coordinates
                \begin{equation}
                        v(\bm{r},\bm{p},\bm{\sigma},\bm{\sigma'},\bm{\tau},\bm{\tau'}) = v(-\bm{r},-\bm{p},\bm{\sigma},\bm{\sigma'},\bm{\tau},\bm{\tau'})
                \end{equation}

        \item \textbf{Time reversal invariance:} Finally, the interaction should stay the same after switching the time arrow direction
                \begin{equation}
                        v(\bm{r},\bm{p},\bm{\sigma},\bm{\sigma'},\bm{\tau},\bm{\tau'}) = v(\bm{r},-\bm{p},-\bm{\sigma},-\bm{\sigma'},\bm{\tau},\bm{\tau'})
                \end{equation}
\end{itemize}
Having the above considerations, developing further the M3Y-Paris interaction \cite{nakada2002method, nakada2003hartree, nakada2010erratum}
\begin{equation}
        v = v_{00}(r) + v_{10}(r) \bm{\sigma}\cdot\bm{\sigma'} + v_{01}(r) \bm{\tau}\cdot\bm{\tau'} + v_{11}(r) (\bm{\sigma}\cdot\bm{\sigma'})(\bm{\tau}\cdot\bm{\tau'})
\end{equation}
by adding a density-dependent form factor to each term gives the CDM3Y$n$ interaction model
\begin{IEEEeqnarray*}{rCl}
        v(\rho,r) &=& F_{00}(\rho) v_{00}(r) + F_{10}(\rho) v_{10}(r) \bm{\sigma}\cdot\bm{\sigma'}\\
          &&\negmedspace{}+ F_{01}(\rho) v_{01}(r) \bm{\tau}\cdot\bm{\tau'} + F_{11}(\rho) v_{11}(r) (\bm{\sigma}\cdot\bm{\sigma'})(\bm{\tau}\cdot\bm{\tau'})\IEEEyesnumber
          \label{eq2-11}
\end{IEEEeqnarray*}  
where each radial term is the superposition of 3 Yukawa potentials with different parameters
\begin{equation}
        v_{\sigma\tau}(r) = \sum^{3}_{k=1} Y_{\sigma\tau}(k) \frac{\exp(-\mu_k r)}{\mu_k r} 
\end{equation}
and the form factor is adapted the form \cite{khoa1997nuclear,tan2020spin,tan2021equation}
\begin{equation}
        F_{\sigma\tau}(\rho) = C_{\sigma\tau} [1 + \alpha_{\sigma\tau} \exp(-\beta_{\sigma\tau}\rho) + \gamma_{\sigma\tau}\rho]
\end{equation}

\section{Equation of States of Nuclear Matter}%
\label{sec:equation_of_states_of_nuclear_matter}

In \gls{HF} formalism, the total \gls{HF} energy of the system can be expressed as
\begin{IEEEeqnarray*}{rCl}
        E &=& \sum^{}_{\sigma\tau} \sum^{k_F^{\sigma\tau}}_{\bm{k}} \frac{\hbar^2 k^2}{2m_\tau} + \frac{1}{2} \sum^{}_{\bm{k}\sigma\tau} \sum^{}_{\bm{k'}\sigma'\tau'} \left[ \matrixel{\bm{k}\sigma\tau,\bm{k'}\sigma'\tau'}{v^D}{\bm{k}\sigma\tau,\bm{k'}\sigma'\tau'} \right.\\
          && \left. \negmedspace{} + \matrixel{\bm{k}\sigma\tau,\bm{k'}\sigma'\tau'}{v^{EX}}{\bm{k'}\sigma\tau,\bm{k}\sigma'\tau'} \right]\IEEEyesnumber
          \label{eqE}
\end{IEEEeqnarray*}  
where the single-particle wave function is plane wave
\begin{equation}
        \ket{\bm{k}\sigma\tau} = \frac{e^{i\bm{k}\cdot\bm{r}}}{ \sqrt{ \Omega}  } \chi_\sigma \chi_\tau
\end{equation}
$\Omega$ being the spatial volume of the system, $k_F^{\sigma\tau} = (6\pi^2 \rho_{\sigma\tau})^{1/3}$ is the Fermi momentum corresponding to spin $\sigma$ and isospin $\tau$, $v^{D(EX)}$ is the direct (exchange) part of the interaction determined from the singlet and triplet-even (and odd) of the central \gls{NN} force. Adopting the same functional form of \eqref{eq2-11}, the direct and exchange interaction is written as
\begin{IEEEeqnarray*}{rCl}
        v^{D(EX)}(\rho_b,r) &=& F^{D(EX)}_{00}(\rho_b) v^{D(EX)}_{00}(r) + F^{D(EX)}_{10}(\rho_b) v^{D(EX)}_{10}(r) \bm{\sigma}\cdot\bm{\sigma'}\\
                          &&\negmedspace{}+ F^{D(EX)}_{01}(\rho_b) v^{D(EX)}_{01}(r) \bm{\tau}\cdot\bm{\tau'}\\
                          &&\negmedspace{}+ F^{D(EX)}_{11}(\rho_b) v^{D(EX)}_{11}(r) (\bm{\sigma}\cdot\bm{\sigma'})(\bm{\tau}\cdot\bm{\tau'})\IEEEyesnumber
                          \label{eqHF}
\end{IEEEeqnarray*}
and
\begin{IEEEeqnarray*}{rCl}
        v^{D(EX)}_{\sigma\tau}(r) &=& \sum^{3}_{k=1} Y^{D(EX)}_{\sigma\tau}(k) \frac{\exp(-\mu_k r)}{\mu_k r} \IEEEyesnumber
\end{IEEEeqnarray*}  
with the Yukawa strengths given in Table \ref{tab:yukawa} and the density-dependent form factor parameters are in Table \ref{tab:cd}. Note that in \eqref{eqHF}, $\rho_b$ denotes the \emph{baryon density}, this will be used in order to distinguish with the lepton density in the later section.

\begin{table}[H]
        \centering
        \caption{CDM3Y$n$ interaction's parameters \cite{tan2020spin,than2010ufr}. The version CDM3Y3, 4, 6 don't have the 10 and 11 terms due to them being \emph{spin saturated}, i.e. the spin dependent terms vanishes effectively.}
        \label{tab:cd}
        \begin{tabular}{|C|C|C|C|C|C|}
                \hline
                \text{Interaction} & \sigma\tau & C_{\sigma\tau} & \alpha_{\sigma\tau} & \beta_{\sigma\tau} & \gamma_{\sigma\tau}\\
                            & & & & (fm^3) & (fm^3)\\
                \hline
                \multirow{2}{*}{\text{CDM3Y3}} & 00 & 0.2985 & 3.4528 & 2.6388 & -1.5\\
                                        & 01 & 0.1574 & 9.7016 & 16.2704 & 11.9946\\
                \hline
                \multirow{2}{*}{\text{CDM3Y4}} & 00 & 0.3052 & 3.2998 & 2.3180 & -2.0\\
                                        & 01 & 0.1318 & 11.7739 & 16.0279 & 15.1987\\
                \hline
                \multirow{2}{*}{\text{CDM3Y6}} & 00 & 0.2658 & 3.8033 & 1.4099 & -4.0\\
                                        & 01 & 0.1824 & 8.8819 & 16.4346 & 10.8703\\
                \hline
                \multirow{4}{*}{\text{CDM3Y8}} & 00 & 0.2658 & 3.8033 & 1.4099 & -4.3\\
                                        & 01 & 0.2463 & 6.3836 & 10.2566 & 6.3549\\
                                        & 10 & 0.2161 & 3.7510 & -3.3396 & 9.9329\\
                                        & 11 & 0.7572 & 1.9967 & 33.2012 & 0.2989\\
                \hline
        \end{tabular}
\end{table}

\begin{table}[H]
        \centering
        \caption{Yukawa strengths of the M3Y-Paris interaction \cite{tan2020spin,anantaraman1983effective}.}
        \label{tab:yukawa}
        \begin{tabular}{|C|C|C|C|C|C|C|C|C|C|}
                \hline
                k & \mu_k & Y^D_{00} & Y^D_{10} & Y^D_{01} & Y^D_{11} & Y^{EX}_{00} & Y^{EX}_{10} & Y^{EX}_{01} & Y^{EX}_{11}\\
                    & (fm^{-1}) & (MeV) & (MeV) & (MeV) & (MeV) & (MeV) & (MeV) & (MeV) & (MeV)\\
                \hline
                1 & 4.0& 11061.625 & 938.875 & 313.625 & -969.125 & -1524.25 & -3492.75 & -4118.0 & -2210.0\\
                2 & 2.5 & -2537.5 & -36.0 & 223.5 & 450.0 & -518.75 & 795.25 & 1054.75 & 568.75\\
                3 & 0.7072 & 0.0 & 0.0 & 0.0 & 3.4877 & -7.8474 & 2.6157 & 2.6157 & -0.8719\\
                \hline
        \end{tabular}
\end{table}
Multiply \eqref{eqE} with $\Omega^{-1}$, the energy density of the \gls{NM} is separated into the kinetic term $\varepsilon_{kin}$ and the potential terms $\varepsilon_{\sigma\tau}$, i.e.
\begin{equation}
        \varepsilon = \frac{E}{\Omega} = \varepsilon_{kin} + F_{00}(\rho_b) \varepsilon_{00} + F_{01}(\rho_b) \varepsilon_{01} + F_{10}(\rho_b) \varepsilon_{10} + F_{11}(\rho_b) \varepsilon_{11}
\end{equation}
The final expressions of each terms of the energy density are
\begin{IEEEeqnarray}{rCl}
        \varepsilon_{kin} &=& \frac{3}{10} \sum^{}_{\sigma\tau} \frac{\hbar^2 (k_F^{\sigma\tau})^2}{m_\tau} \rho_{\sigma\tau}\\
        \varepsilon_{00} &=& \frac{1}{2} \left[ \rho_b^2 J^D_{00} + \int {A^2_{00} v^{EX}_{00}(r)} \: d^3 r \right] \\
        \varepsilon_{10} &=& \frac{1}{2} \left[ \rho_b^2 J^D_{10} \left( \Delta_n \frac{1+\delta}{2} + \Delta_p \frac{1-\delta}{2} \right)^2 + \int {A^2_{10} v^{EX}_{10}(r)} \: d^3 r \right] \\
        \varepsilon_{01} &=& \frac{1}{2} \left[ \rho_b^2 J^D_{01}\delta^2 + \int {A^2_{01} v^{EX}_{01}(r)} \: d^3 r \right] \\
        \varepsilon_{11} &=& \frac{1}{2} \left[ \rho_b^2 J^D_{11} \left( \Delta_n \frac{1+\delta}{2} - \Delta_p \frac{1-\delta}{2} \right)^2 + \int {A^2_{11} v^{EX}_{11}(r)} \: d^3 r \right]
\end{IEEEeqnarray}  
where $\Delta_{\tau} = (\rho_{\uparrow \tau} - \rho_{\downarrow \tau})/\rho_{\tau}$ is the polarity of nucleon, $\delta = (\rho_n - \rho_p)/\rho_b$ is the asymmetry of \gls{NM}, $J^D_{\sigma\tau} = \int v_{\sigma\tau}(r)\: d^3 r$ is the volume integral of the direct interaction and
\begin{equation}
        \begin{array}{l}
                A_{00} = \rho_{\uparrow n} \hat{j_1}(k_F^{\uparrow n}) + \rho_{\downarrow n} \hat{j_1}(k_F^{\downarrow n}) + \rho_{\uparrow p} \hat{j_1}(k_F^{\uparrow p}) + \rho_{\downarrow p} \hat{j_1}(k_F^{\downarrow p})\\[5pt]
                A_{10} = \rho_{\uparrow n} \hat{j_1}(k_F^{\uparrow n}) - \rho_{\downarrow n} \hat{j_1}(k_F^{\downarrow n}) + \rho_{\uparrow p} \hat{j_1}(k_F^{\uparrow p}) - \rho_{\downarrow p} \hat{j_1}(k_F^{\downarrow p})\\[5pt]
                A_{01} = \rho_{\uparrow n} \hat{j_1}(k_F^{\uparrow n}) + \rho_{\downarrow n} \hat{j_1}(k_F^{\downarrow n}) - \rho_{\uparrow p} \hat{j_1}(k_F^{\uparrow p}) - \rho_{\downarrow p} \hat{j_1}(k_F^{\downarrow p})\\[5pt]
                A_{11} = \rho_{\uparrow n} \hat{j_1}(k_F^{\uparrow n}) - \rho_{\downarrow n} \hat{j_1}(k_F^{\downarrow n}) - \rho_{\uparrow p} \hat{j_1}(k_F^{\uparrow p}) + \rho_{\downarrow p} \hat{j_1}(k_F^{\downarrow p})
        \end{array}
\end{equation}
with $\hat{j}_1(x)=3j_1(x)/x$ and $j_1(x)$ being the 1\textsuperscript{st} order spherical Bessel function.

\section{\textbeta-Stable Nuclear Matter}%
\label{sec:textbeta_stable_nuclear_matter}

After the \gls{HF} calculation, we were able to obtain a numerical \gls{HF} energy density $\varepsilon(\rho_n,\rho_p,\Delta_n,\Delta_p)$. However, it is in fact impossible for a \gls{NS} to exist while consisting of purely nucleon. In order for the \gls{NS} to exist, leptons ($e^-$ and $\mu^-$) have to be introduced to the matter constituents and the $npe\mu$ matter has to be under the condition of \emph{\textbeta-stable}, i.e.
\begin{itemize}
        \item Charge balance
                \begin{equation}
                        \rho_p = \rho_e + \rho_\mu
                        \label{chargeEQ}
                \end{equation}
        \item Chemical potential balance
                \begin{equation}
                        \mu_n - \mu_p = \mu_e = \mu_\mu
                \end{equation}
                where $\mu_i$ ($i=n,p,e,\mu$) is the chemical potential of the $i$ particle.
\end{itemize}
The total energy density of the $npe\mu$ matter is thus
\begin{equation}
        \varepsilon = \varepsilon_{HF} + \rho_n m_n c^2 + \rho_p m_p c^2 + \varepsilon_e + \varepsilon_\mu 
\end{equation}
which leads to the nucleon chemical potential of the form
\begin{equation}
        \mu_\tau (\rho_n,\rho_p,\Delta_n,\Delta_p) = \frac{\partial \varepsilon}{\partial \rho_\tau}  = \frac{\partial \varepsilon_{HF}}{\partial \rho_\tau} + m_\tau c^2
\end{equation}
Let $\hat{\mu} = \mu_n - \mu_p$ be the leptons' chemical potential, \eqref{chargeEQ} is equivalent to\footnote{$\theta(x)$ is the Heaviside function, i.e. it returns $1$ for $x\geq 0$ and $0$ otherwise.}
\begin{equation}
        3\pi^2 (\hbar c)^3 \rho_p - \hat{\mu}^3 - \left[ \hat{\mu}^2 - (m_\mu c^2)^2 \right]^{3/2} \theta(\hat{\mu} - m_\mu c^2) = 0
\end{equation}
In the CDM3Y3, 4, 6 interactions, the \gls{NM} is spin saturated, therefore a $\Delta_n = \Delta_p = 0$ and there are no 10 and 11 terms, while for the interaction CDM3Y8, under strong magnetic field like that of a magnetar, we can approximate $\Delta_n \approx -\Delta_p \approx \Delta$ and reduce the \gls{EoS} to depend on just the baryon polarity $\Delta$ alone, and the more baryon polarized, the stronger the magnetic field of the \gls{NS}.\par
For a fixed value of $\Delta$, we are able to obtain a density function of the form $\rho_n (\rho_b, \Delta)$ and $\rho_p (\rho_b, \Delta)$, which in turn gives the lepton chemical potential $\hat{\mu}(\rho_b,\Delta) = \hat{\mu}(\rho_n,\rho_p)$ On the other hand, the leptons' densities are then
\begin{equation}
        \rho_e(\rho_b,\Delta) = \frac{ \hat{\mu}^3(\rho_b,\Delta)}{ 3\pi^2 (\hbar c)^3} \quad\text{and}\quad \rho_\mu(\rho_b,\Delta) = \frac{ \Big[\hat{\mu}^2(\rho_b,\Delta) - (m_\mu c^2)^2\Big]^{3/2}}{ 3\pi^2 (\hbar c)^3} \theta(\hat{\mu}(\rho_b,\Delta)-m_\mu c^2)
\end{equation} 
Consider the $e^-$ and $\mu^-$ to be systems of relativistic Fermi gas, then their respective energy densities and pressure contributions are ($l=e,\mu$)
\begin{equation}
        \varepsilon_l(\rho_b,\Delta) = \frac{ 2}{ (2\pi)^3} \int_{{0}}^{{[3\pi^2\rho_l(\rho_b,\Delta)]^{1/3}}} {\sqrt{\hbar^2 c^2 k^2 + m_l^2 c^4}} \: d^3{\mathbf{k}}
\end{equation} 
and
\begin{equation}
        P_l(\rho_b,\Delta) = \frac{ 1}{ 3} \frac{ 2}{ (2\pi)^3} \int_{{0}}^{{[3\pi^2 \rho_l(\rho_b,\Delta)]^{1/3}}} { \frac{ \hbar^2 c^2 k^2}{ \sqrt{\hbar^2 c^2 k^2 + m_l^2 c^4}} } \: d^3{\mathbf{k}}
\end{equation} 
Plus, from the \gls{HF} formalism with \gls{NM}, the baryon pressure is given by
\begin{equation}
        P_b = \rho_b^2 \pdv{(\varepsilon_{HF}/\rho_b)}{\rho_b}
\end{equation}
Finally, we obtain the total energy density dependence on baryon density as 
\begin{equation}
        \varepsilon(\rho_b,\Delta) = \varepsilon_{HF}(\rho_b,\Delta) + \rho_n(\rho_b,\Delta)m_n c^2 + \rho_p(\rho_b,\Delta)m_p c^2 + \varepsilon_e(\rho_b,\Delta) + \varepsilon_\mu(\rho_b,\Delta)
\end{equation}
and the total pressure of \gls{NS} matter
\begin{equation}
        P(\rho_b,\Delta) = P_b(\rho_b,\Delta) + P_e(\rho_b,\Delta) + P_\mu(\rho_b,\Delta)
\end{equation}
and this is the final result of the \gls{EoS} of cold $\beta$-stable \gls{NS} matter.
