\section{Nucleon-Nucleon Interaction}%
\label{sec:nucleon_nucleon_interaction}

Due to the lack of a exact theory to describe the nucleon-nucleon (\gls{NN}) interaction, a model need to be imposed and fit with experimental measurement or theoretical calculation results. Plus, for a system as massive as a \gls{NS}, deducing the \gls{EoS} using the \emph{ab initio} method, i.e. solving the Schr\"{o}dinger equation over all particles, is simply impossible, therefore an \emph{effective interaction} must be used \citep{greiner1996nuclear}. In this section, we only limit ourselves to two-body interaction, thus, the \gls{NN} potential can be expressed in the form of
\begin{equation}
        v = v(\bm{r}, \bm{r'}, \bm{p}, \bm{p'}, \bm{\sigma}, \bm{\sigma'}, \bm{\tau}, \bm{\tau'})
        \label{eff_interaction}
\end{equation}
where the primed and unprimed variables indicate the properties of 2 nucleons respectively, in which $\bm{r}$ is the particle's position, $\bm{p}$ is its momentum, $\bm{\sigma}$ is its intrinsic spin and $\bm{\tau}$ is its isospin.\par
The functional form of $v$ in \eqref{eff_interaction} cannot freely take any form but is constrained by many invariance requirements \citep{greiner1996nuclear}, namely the translational, Galilei, rotational, isospin, parity and time reversal invariance. Having such considerations, developing further the M3Y-Paris interaction, which was used by the \gls{HF} study of \gls{NM} \citep{loan2011equation, tan2016mean, tan2020spin,tan2021equation} and the folding model study of \gls{NN} scattering \citep{khoa1997nuclear,khoa2000generalized},
\begin{equation}
        v = v_{00}(r) + v_{10}(r) \bm{\sigma}\cdot\bm{\sigma'} + v_{01}(r) \bm{\tau}\cdot\bm{\tau'} + v_{11}(r) (\bm{\sigma}\cdot\bm{\sigma'})(\bm{\tau}\cdot\bm{\tau'})
\end{equation}
by adding a density-dependent form factor $F_{\sigma\tau}(n)$ to each term gives the CDM3Y$n$ interaction model
\begin{IEEEeqnarray*}{rCl}
        v(n,r) &=& F_{00}(n) v_{00}(r) + F_{10}(n) v_{10}(r) \bm{\sigma}\cdot\bm{\sigma'}\\
          &&\negmedspace{}+ F_{01}(n) v_{01}(r) \bm{\tau}\cdot\bm{\tau'} + F_{11}(n) v_{11}(r) (\bm{\sigma}\cdot\bm{\sigma'})(\bm{\tau}\cdot\bm{\tau'})\IEEEyesnumber
          \label{eq2-11}
\end{IEEEeqnarray*}  
where each radial term is the superposition of 3 Yukawa potentials
\begin{equation}
        v_{\sigma\tau}(r) = \sum^{3}_{k=1} Y_{\sigma\tau}(k) \frac{\exp(-\mu_k r)}{\mu_k r} 
\end{equation}
and the form factor $F_{\sigma\tau}(n)$ shared the functional form \citep{khoa1997nuclear,tan2020spin,tan2021equation,than2010ufr}
\begin{equation}
        F_{\sigma\tau}(n) = C_{\sigma\tau} [1 + \alpha_{\sigma\tau} \exp(-\beta_{\sigma\tau}n) + \gamma_{\sigma\tau}n]
\end{equation}
with parameters given in Table \ref{tab:cd}. The parameters of $F_{00}$ were adjusted to give the corresponding incompressibility $K$ of symmetric \gls{NM} at saturation density $n_0$ and the binding energy $E_0 \approx 15.8\; MeV$, while the 10 term is modified from \citep{than2010ufr} to reproduce $E_{sym}(n_0) \approx 30\;MeV$, $L\approx 50\;MeV$ and to be in agreement with the ab-initio results \citep{akmal1998equation,gandolfi2010microscopic} at higher density \citep{tan2021equation}. On the other hand, the spin-dependent terms, 10 and 11, are hereby included in the 5 models by fine tuning the parameters to yield the same result as the Brueckner-Hartree-Fock (\gls{BHF}) study of spin polarized \gls{NM} \citep{vidana2002equation} as in Figure \ref{fig:bhf}.

\begin{figure}[ht]
        \centering
        \includegraphics[width=\textwidth]{fig/BHF_fit.eps}
        \caption{Energy per baryon $E/A$ of symmetric \gls{NM} by the 5 CDM3Y$n$ models compared to \gls{BHF} result \citep{vidana2002equation}. The diamond and square represent the \gls{BHF} result for $\Delta_n=-\Delta_p=\pm 1$ and $\Delta_n=\Delta_p=\pm 1$ respectively with $\Delta_\tau$ being the baryon spin polarization.}
        \label{fig:bhf}
\end{figure} 

\begin{table}[ht]
        \centering
        \caption{CDM3Y$n$ interaction's parameters; the 00 and 01 terms are inherited from \citep{tan2021equation}, while the 10 and 11 parameters are added by fitting with \gls{BHF} result and $K$ is the incompressibility of \gls{NM}.}
        \label{tab:cd}
        \begin{tabular}{|C|C|C|C|C|C|C|}
                \hline
                \text{Interaction} & \sigma\tau & C_{\sigma\tau} & \alpha_{\sigma\tau} & \beta_{\sigma\tau} & \gamma_{\sigma\tau} & K\\
                                   & & & & (fm^3) & (fm^3) & (MeV)\\
                \hline
                \multirow{4}{*}{\text{CDM3Y3}} & 00 & 0.2985 & 3.4528 & 2.6388 & -1.5 &\multirow{4}{*}{217}\\
                                               & 01 & 0.2343 & 5.3336 & 6.4738 & 4.3172 &\\
                                               & 10 & 0.3890 & 3.5635 & -2.6717 & 20.3624 &\\
                                               & 11 & 0.8802 & 4.0433 & 12.3262 & 0.3662 &\\
                \hline
                \multirow{4}{*}{\text{CDM3Y4}} & 00 & 0.3052 & 3.2998 & 2.3180 & -2.0 &\multirow{4}{*}{228}\\
                                               & 01 & 0.2129 & 6.3581 & 7.0584 & 5.6091 &\\
                                               & 10 & 0.2593 & 6.0016 & -2.3377 & 18.8725 &\\
                                               & 11 & 0.8329 & 3.5941 & 9.2012 & 0.2690 &\\
                \hline
                \multirow{4}{*}{\text{CDM3Y5}} & 00 & 0.2728 & 3.7367 & 1.8294 & -3.0 &\multirow{4}{*}{241}\\
                                               & 01 & 0.2204 & 6.6146 & 7.9910 & 6.0040 &\\
                                               & 10 & 0.4106 & 5.6265 & -1.6698 & -1.9866 &\\
                                               & 11 & 0.6815 & 2.5833 & 5.1700 & 0.2578 &\\
                \hline
                \multirow{4}{*}{\text{CDM3Y6}} & 00 & 0.2658 & 3.8033 & 1.4099 & -4.0 &\multirow{4}{*}{252}\\
                                               & 01 & 0.2313 & 6.6865 & 8.6775 & 6.0182 &\\
                                               & 10 & 0.5186 & 9.9402 & 1.6698 & 2.9799 &\\
                                               & 11 & 0.6058 & 3.1947 & 4.4512 & 0.0822 &\\
                \hline
                \multirow{4}{*}{\text{CDM3Y8}} & 00 & 0.2658 & 3.8033 & 1.4099 & -4.3 &\multirow{4}{*}{257}\\
                                               & 01 & 0.2643 & 6.3836 & 9.8950 & 5.4249 &\\
                                               & 10 & 0.5997 & 9.1900 & 0.7514 & -4.7181 &\\
                                               & 11 & 0.3786 & 3.9435 & 2.7012 & 0.3512 &\\
                \hline
        \end{tabular}
\end{table}

\section{Equation of States of Nuclear Matter}%
\label{sec:equation_of_states_of_nuclear_matter}

In \gls{HF} formalism, the total \gls{HF} energy of the system can be expressed as
\begin{IEEEeqnarray*}{rCl}
        E_{HF} &=& \sum^{}_{\sigma\tau} \sum^{k_F^{\sigma\tau}}_{\bm{k}} \frac{\hbar^2 k^2}{2m_\tau} + \frac{1}{2} \sum^{}_{\bm{k}\sigma\tau} \sum^{}_{\bm{k'}\sigma'\tau'} \left[ \mel**{\bm{k}\sigma\tau,\bm{k'}\sigma'\tau'}{v^D}{\bm{k}\sigma\tau,\bm{k'}\sigma'\tau'} \right.\\
          && \left. \negmedspace{} + \mel**{\bm{k}\sigma\tau,\bm{k'}\sigma'\tau'}{v^{EX}}{\bm{k'}\sigma\tau,\bm{k}\sigma'\tau'} \right]\IEEEyesnumber
          \label{eqE}
\end{IEEEeqnarray*}  
where the single-particle wave function is plane wave
\begin{equation}
        \ket{\bm{k}\sigma\tau} = \frac{e^{i\bm{k}\cdot\bm{r}}}{ \sqrt{ \Omega}  } \chi_\sigma \chi_\tau
\end{equation}
$\Omega$ being the spatial volume of the system, $k_F^{\sigma\tau} = (6\pi^2 n_{\sigma\tau})^{1/3}$ is the Fermi momentum corresponding to spin $\sigma$ and isospin $\tau$, $v^{D(EX)}$ is the direct (exchange) part of the interaction determined from the singlet and triplet-even (odd) of the central \gls{NN} force. Adopting the same functional form of \eqref{eq2-11}, the direct and exchange interaction is written as
\begin{IEEEeqnarray*}{rCl}
        v^{D(EX)}(n_b,r) &=& F_{00}(n_b) v^{D(EX)}_{00}(r) + F_{10}(n_b) v^{D(EX)}_{10}(r) \bm{\sigma}\cdot\bm{\sigma'}\\
                          &&\negmedspace{}+ F_{01}(n_b) v^{D(EX)}_{01}(r) \bm{\tau}\cdot\bm{\tau'} + F_{11}(n_b) v^{D(EX)}_{11}(r) (\bm{\sigma}\cdot\bm{\sigma'})(\bm{\tau}\cdot\bm{\tau'})\IEEEyesnumber
                          \label{eqHF}
\end{IEEEeqnarray*}
and
\begin{IEEEeqnarray*}{rCl}
        v^{D(EX)}_{\sigma\tau}(r) &=& \sum^{3}_{k=1} Y^{D(EX)}_{\sigma\tau}(k) \frac{\exp(-\mu_k r)}{\mu_k r} \IEEEyesnumber
\end{IEEEeqnarray*}  
with the Yukawa strengths given in Table \ref{tab:yukawa} and the density-dependent form factor parameters are in Table \ref{tab:cd}. Note that in \eqref{eqHF}, $n_b$ denotes the \emph{baryon density}, this will be used in order to distinguish with the lepton density in the later section.

\begin{table}[H]
        \centering
        \caption{Yukawa strengths of the M3Y-Paris interaction \citep{tan2020spin,anantaraman1983effective}.}
        \label{tab:yukawa}
        \begin{tabular}{|C|C|C|C|C|C|C|C|C|C|}
                \hline
                k & \mu_k & Y^D_{00} & Y^D_{10} & Y^D_{01} & Y^D_{11} & Y^{EX}_{00} & Y^{EX}_{10} & Y^{EX}_{01} & Y^{EX}_{11}\\
                    & (fm^{-1}) & (MeV) & (MeV) & (MeV) & (MeV) & (MeV) & (MeV) & (MeV) & (MeV)\\
                \hline
                1 & 4.0& 11061.625 & 938.875 & 313.625 & -969.125 & -1524.25 & -3492.75 & -4118.0 & -2210.0\\
                2 & 2.5 & -2537.5 & -36.0 & 223.5 & 450.0 & -518.75 & 795.25 & 1054.75 & 568.75\\
                3 & 0.7072 & 0.0 & 0.0 & 0.0 & 3.4877 & -7.8474 & 2.6157 & 2.6157 & -0.8719\\
                \hline
        \end{tabular}
\end{table}
Multiply \eqref{eqE} with $\Omega^{-1}$, the energy density of the \gls{NM} is separated into the kinetic term $\varepsilon_{kin}$ and the potential terms $\varepsilon_{\sigma\tau}$, i.e.
\begin{equation}
        \varepsilon_{HF} = \frac{E_{HF}}{\Omega} = \varepsilon_{kin} + F_{00}(n_b) \varepsilon_{00} + F_{01}(n_b) \varepsilon_{01} + F_{10}(n_b) \varepsilon_{10} + F_{11}(n_b) \varepsilon_{11}
\end{equation}
The final expressions of each terms of the energy density are
\begin{IEEEeqnarray}{rCl}
        \varepsilon_{kin} &=& \frac{3}{10} \sum^{}_{\sigma\tau} \frac{\hbar^2 (k_F^{\sigma\tau})^2}{m_\tau} n_{\sigma\tau}\\
        \varepsilon_{00} &=& \frac{1}{2} \left[ n_b^2 J^D_{00} + \int {A^2_{00} v^{EX}_{00}(r)} \: d^3 r \right] \\
        \varepsilon_{10} &=& \frac{1}{2} \left[ n_b^2 J^D_{10} \left( \Delta_n \frac{1+\delta}{2} + \Delta_p \frac{1-\delta}{2} \right)^2 + \int {A^2_{10} v^{EX}_{10}(r)} \: d^3 r \right] \\
        \varepsilon_{01} &=& \frac{1}{2} \left[ n_b^2 J^D_{01}\delta^2 + \int {A^2_{01} v^{EX}_{01}(r)} \: d^3 r \right] \\
        \varepsilon_{11} &=& \frac{1}{2} \left[ n_b^2 J^D_{11} \left( \Delta_n \frac{1+\delta}{2} - \Delta_p \frac{1-\delta}{2} \right)^2 + \int {A^2_{11} v^{EX}_{11}(r)} \: d^3 r \right]
\end{IEEEeqnarray}  
where $\Delta_{\tau} = (n_{\uparrow \tau} - n_{\downarrow \tau})/n_{\tau}$ is the polarization of nucleon, $\delta = (n_n - n_p)/n_b$ is the asymmetry of \gls{NM}, $J^D_{\sigma\tau} = \int v^D_{\sigma\tau}(r)\: d^3 r$ is the volume integral of the direct interaction and
\begin{equation}
        \begin{array}{l}
                A_{00} = n_{\uparrow n} \hat{j_1}(k_F^{\uparrow n} r) + n_{\downarrow n} \hat{j_1}(k_F^{\downarrow n} r) + n_{\uparrow p} \hat{j_1}(k_F^{\uparrow p} r) + n_{\downarrow p} \hat{j_1}(k_F^{\downarrow p} r)\\[5pt]
                A_{10} = n_{\uparrow n} \hat{j_1}(k_F^{\uparrow n} r) - n_{\downarrow n} \hat{j_1}(k_F^{\downarrow n} r) + n_{\uparrow p} \hat{j_1}(k_F^{\uparrow p} r) - n_{\downarrow p} \hat{j_1}(k_F^{\downarrow p} r)\\[5pt]
                A_{01} = n_{\uparrow n} \hat{j_1}(k_F^{\uparrow n} r) + n_{\downarrow n} \hat{j_1}(k_F^{\downarrow n} r) - n_{\uparrow p} \hat{j_1}(k_F^{\uparrow p} r) - n_{\downarrow p} \hat{j_1}(k_F^{\downarrow p} r)\\[5pt]
                A_{11} = n_{\uparrow n} \hat{j_1}(k_F^{\uparrow n} r) - n_{\downarrow n} \hat{j_1}(k_F^{\downarrow n} r) - n_{\uparrow p} \hat{j_1}(k_F^{\uparrow p} r) + n_{\downarrow p} \hat{j_1}(k_F^{\downarrow p} r)
        \end{array}
\end{equation}
with $\hat{j}_1(x)=3j_1(x)/x$ and $j_1(x)$ being the 1\textsuperscript{st} order spherical Bessel function.\par
In the parabolic approximation \citep{khoa1996study}, the energy density per nucleon $E/A$ can also be expanded in terms of the asymmetry $\delta$ as
\begin{equation}
    \frac{E}{A} (n_b, \delta, \Delta_n, \Delta_p) = \frac{\varepsilon_{HF}}{n_b} = \frac{E}{A} (n_b, \delta=0, \Delta_n, \Delta_p) + S(n_b, \Delta_n, \Delta_p)\delta^2 + \mathcal{O}(\delta^4)
\end{equation}
with $S$ being the \emph{nuclear symmetry energy}. The symmetry coefficient $J$, slope parameter $L$ and curvature $K_{sym}$ are taken by expanding the symmetry energy at saturation density $n_0$, i.e. \citep{tan2020spin,li2008recent,horowitz2014way,lattimer2014symmetry}
\begin{equation}
    S(n_b, \Delta_n, \Delta_p) = J(\Delta_n, \Delta_p) + \frac{L(\Delta_n, \Delta_p)}{3} \left( \frac{n_b - n_0}{n_0} \right) + \frac{K_{sym}(\Delta_n, \Delta_p)}{18} \left( \frac{n_b - n_0}{n_0} \right)^2 + \ldots
\end{equation}
along with the nuclear incompressibility at saturation density
\begin{equation}
    K(\Delta_n, \Delta_p) = 9 \left.\pdv{P(n_b, \delta=0,\Delta_n,\Delta_p)}{n_b}\right|_{n_b \to n_0}
\end{equation}
These are the quantity that will be used in order to compared with the empirical values in Chapter \ref{chap:result}.

\section{\textbeta-Stable Nuclear Matter}%
\label{sec:textbeta_stable_nuclear_matter}

After the \gls{HF} calculation, we were able to obtain a numerical \gls{HF} energy density $\varepsilon(n_n,n_p,\Delta_n,\Delta_p)$. However, it is in fact impossible for a \gls{NS} to exist while consisting of purely nucleon. In order to compensate for this issue, leptons ($e^-$ and $\mu^-$) have to be introduced to the matter constituents and the $npe\mu$ matter has to satisfy the \emph{\textbeta-stable} condition \citep{glendenning2012compact}, i.e.
\begin{itemize}
        \item Charge balance
                \begin{equation}
                        n_p = n_e + n_\mu
                        \label{chargeEQ}
                \end{equation}
        \item Chemical potential balance
                \begin{equation}
                        \mu_n - \mu_p = \mu_e = \mu_\mu
                \end{equation}
                where $\mu_i = \pdv{\varepsilon}{n_i}$ ($i=n,p,e,\mu$) is the chemical potential of the $i$ particle.
\end{itemize}
The total energy density of the $npe\mu$ matter is thus
\begin{equation}
        \varepsilon = \varepsilon_{HF} + n_n m_n c^2 + n_p m_p c^2 + \varepsilon_e + \varepsilon_\mu 
\end{equation}
which leads to the nucleon chemical potential of the form
\begin{equation}
        \mu_\tau (n_n,n_p,\Delta_n,\Delta_p) = \frac{\partial \varepsilon}{\partial n_\tau}  = \frac{\partial \varepsilon_{HF}}{\partial n_\tau} + m_\tau c^2
\end{equation}
Let $\hat{\mu} = \mu_n - \mu_p$ be the leptons' chemical potential, \eqref{chargeEQ} is equivalent to\footnote{$\theta(x)$ is the Heaviside function, i.e. it returns $1$ for $x\geq 0$ and $0$ otherwise.}
\begin{equation}
        3\pi^2 (\hbar c)^3 n_p - \hat{\mu}^3 - \left[ \hat{\mu}^2 - (m_\mu c^2)^2 \right]^{3/2} \theta(\hat{\mu} - m_\mu c^2) = 0
\end{equation}
from which the proton fraction $x_p = n_p/n_b$ can be obtained as shown in Figure \ref{fig:xp}, note that only beyond the muon threshold density $\mu_e > m_\mu c^2 \approx 105.6\:MeV$ do muons appear in the system. Furthermore, under strong magnetic field like that of a magnetar, we can approximate $\Delta_n \approx -\Delta_p \approx \Delta$ and reduce the \gls{EoS} to depend on just the baryon polarization $\Delta$ alone, and the more baryon polarized, the stronger the magnetic field of the \gls{NS}.\par
For a fixed value of $\Delta$, we are able to obtain a density function of the form $n_n (n_b, \Delta)$ and $n_p (n_b, \Delta)$, which in turn gives the lepton chemical potential $\hat{\mu}(n_b,\Delta) = \hat{\mu}(n_n,n_p)$ On the other hand, the leptons' densities are then \citep{loan2011equation}
\begin{equation}
        n_e(n_b,\Delta) = \frac{ \hat{\mu}^3(n_b,\Delta)}{ 3\pi^2 (\hbar c)^3} \quad\text{and}\quad n_\mu(n_b,\Delta) = \frac{ \Big[\hat{\mu}^2(n_b,\Delta) - (m_\mu c^2)^2\Big]^{3/2}}{ 3\pi^2 (\hbar c)^3} \theta(\hat{\mu}(n_b,\Delta)-m_\mu c^2)
\end{equation} 
Consider the $e^-$ and $\mu^-$ to be systems of relativistic Fermi gas, then their respective energy densities and pressure contributions are ($l=e,\mu$) \citep{moustakidis2009equation}
\begin{equation}
        \varepsilon_l(n_b,\Delta) = \frac{ 2}{ (2\pi)^3} \int_{{0}}^{{[3\pi^2n_l(n_b,\Delta)]^{1/3}}} {\sqrt{\hbar^2 c^2 k^2 + m_l^2 c^4}} \: d^3{\mathbf{k}}
\end{equation} 
and
\begin{equation}
        P_l(n_b,\Delta) = \frac{ 1}{ 3} \frac{ 2}{ (2\pi)^3} \int_{{0}}^{{[3\pi^2 n_l(n_b,\Delta)]^{1/3}}} { \frac{ \hbar^2 c^2 k^2}{ \sqrt{\hbar^2 c^2 k^2 + m_l^2 c^4}} } \: d^3{\mathbf{k}}
\end{equation} 
Plus, from the \gls{HF} formalism with \gls{NM}, the baryon pressure is given by
\begin{equation}
        P_b = n_b^2 \pdv{(\varepsilon_{HF}/n_b)}{n_b}
\end{equation}
Finally, we obtain the total energy density dependence on baryon density as 
\begin{equation}
        \varepsilon(n_b,\Delta) = \varepsilon_{HF}(n_b,\Delta) + n_n(n_b,\Delta)m_n c^2 + n_p(n_b,\Delta)m_p c^2 + \varepsilon_e(n_b,\Delta) + \varepsilon_\mu(n_b,\Delta)
\end{equation}
and the total pressure of \gls{NS} matter
\begin{equation}
        P(n_b,\Delta) = P_b(n_b,\Delta) + P_e(n_b,\Delta) + P_\mu(n_b,\Delta)
\end{equation}
In addition, the \gls{EoS} of the \gls{NS}'s crust (low baryon density region) is adopted from the Compress Liquid Drop Model calculation \citep{douchin2000nuclear,douchin2001unified}. The complete \gls{EoS} of cold $\beta$-stable \gls{NS} matter is then given in Figure \ref{fig:s} and \ref{fig:p}.
