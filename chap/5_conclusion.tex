By extending the 4 current versions of density-dependent \gls{NN} interaction \citep{tan2021equation}, we have been able to include the description of spin polarization effect of \gls{NS} matter under strong magnetic field over a wider range of $K$ into the \gls{EoS} of \gls{NM} using the nonrelativistic \gls{HF} formalism. In general, concerning the \gls{EoS} of \gls{NM}, the empirical ranges deduced for symmetry energy $S$, energy per baryon of symmetric \gls{NM} $E/A$ and total pressure $P$ seem to favor partial to zero polarization, or in particular, the cases with lower $\Delta$ overall. Among these 4 versions of \gls{EoS}, the CDM3Y4 interaction is very unlikely to occur in reality, as the \gls{EoS} generated by these interactions are usually outside of the 90\% \gls{CFL} boundary obtained from astrophysical observation and analysis at high density.

The density-dependence of the polarization $\Delta$ is also taken into account by using different situations regarding the localization of magnetic field at the surface of magnetar \citep{tan2020spin}, which eventually eliminates the possibility of total polarization at the magnetar's surface, i.e. $\Delta_0\approx 1.0$. The macroscopic properties of the \gls{NS}, on the other hand, only the cases of partial polarization at the scenario A succeed in staying within the GW170817 constraint \citep{abbott2018gw170817}, which supports the suggestion of the blue kilonova ejecta of GW170817 coming from a magnetar \citep{metzger2018magnetar,tan2020spin}. Furthermore, only the CDM3Y6 and CDM3Y8 in these scenarios can come close to satisfy the lower mass limit of the two heaviest pulsars ever observed, i.e. \gls{PSR} J0348+0432 and \gls{PSR} J0740+6620. The tidal deformability parameter $\Lambda_2$, surprisingly, do not provide further constraint, as all cases are within the accepted range obtained by \cite{abbott2018gw170817}. Beside, higher orders of multipoles up to the 4\textsuperscript{th} order have been investigated for both gravito-electric and gravito-magnetic tidal perturbations. The common diminishing effects as the order $\ell$ increases are founded for all cases, as well as the dominance of the \gls{GE} tidal Love numbers in magnitude compared to the \gls{GM} ones, which is in agreement with the contribution of each term in the phase of \gls{GW} signal \citep{abdelsalhin2018post} as only the $\ell=2$ order of GE tidal deformation contributes significantly enough to be detectable by current measurements. The observation of \gls{GM} tidal deformability can be possibly used as an additional test for Einstein's theory of \gls{GR}, since these effects are not presented in Newtonian physics. More information on the \gls{EoS} of \gls{NS} matter can also be further constrained with the potential observation of the higher order terms, which might be possible with the 3\textsuperscript{rd}-generation \gls{GW} detectors \citep{maggiore2020science}.

In conclusion, regarding the research progress made within this Thesis, all of the objectives demanded have been fulfilled. Additionally, there are several ways in which this topic can be further expanded in the future, i.e.
\begin{itemize}
    \item Accurately calculation of the density dependence of $\Delta(n_b)$,
    \item Extending to a model of hot \textbeta-stable \gls{NS} matter ($T>0\:K$),
    \item Describing the quark-matter core in the interior core of \gls{NS}.
\end{itemize}
