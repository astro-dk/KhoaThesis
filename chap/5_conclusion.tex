By extending the 6 current versions of density-dependent \gls{NN} interaction \citep{tan2021equation}, we have been able to include the description of spin polarization effect of \gls{NS} matter under strong magnetic field over a wider range of $K$ into the \gls{EoS} of \gls{NM} using the nonrelativistic \gls{HF} formalism. The density-dependence of the polarization $\Delta$ was also taken into account as different situations regarding the localization of magnetic field at the surface of \gls{NS}, adopting from the study of \cite{tan2020spin}. From these scenarios and \glsplural{EoS}, the case of total polarization at the surface ($\Delta_0 = 1.0$) has been ruled out, as it failed most of the criteria demanded by the empirical constraints. In general, concerning the \gls{EoS} of \gls{NM}, the empirical ranges deduced for symmetry energy $S$, energy per baryon of symmetric \gls{NM} $E/A$ and total pressure $P$ seem to favor partial polarization, or in particular, the cases with lower $\Delta$ overall. Among these 6 versions of \gls{EoS}, the CDM3Y3 and CDM3Y4 interactions are very unlikely to occur in reality, as the \gls{EoS} generated by these interactions are usually outside of the 90\% \gls{CFL} boundary obtained from astrophysical observation and analysis at high density.\par
The macroscopic properties of the \gls{NS}, on the other hand, only the scenarios with $\Delta_0 = 0.6$ and the case of $\Delta_0 = 0.8$ at the scenario A succeed in staying within the GW170817 constraint \citep{abbott2018gw170817}, which supports the suggestion of the blue kilonova ejecta of GW170817 coming from a magnetar \citep{metzger2018magnetar,tan2020spin}. Furthermore, only the CDM3Y6, CDM3Y8 and BDM3Y1 in these scenarios can come close to satisfy the lower mass limit of the two heaviest pulsars ever observed, i.e. \gls{PSR} J0348+0432 and \gls{PSR} J0740+6620. The tidal deformability parameter $\Lambda_2$, surprisingly, do not provide further constraint, as all cases are within the accepted range obtained by \cite{abbott2018gw170817}. Beside, higher orders of multipoles up to the 4\textsuperscript{th} order have been investigated for both gravito-electric and gravito-magnetic tidal perturbations. The common diminishing effects as the order $l$ increases are founded for all cases, as well as the dominance of the \gls{GE} tidal Love numbers in magnitude compared to the \gls{GM} ones, which is understandable with the contribution of each term in the phase of \gls{GW} signal \citep{abdelsalhin2018post}. The observation of \gls{GM} tidal deformability can be very possibly used as an additional test of Einstein's theory of \gls{GR}, since these effects are not presented in Newtonian physics. More information on the \gls{EoS} of \gls{NS} matter can also be further constrained with the potential observation of the higher order terms, which might be possible with the third-generation \gls{GW} detectors.
