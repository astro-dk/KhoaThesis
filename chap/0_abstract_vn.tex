Phương trình trạng thái của vật chất sao neutron có phân cực spin trong từ trường mạnh đã
được nghiên cứu bằng phương pháp trường trung bình trong mẫu Hartree-Fock bằng 4 tương tác
hiệu dụng phụ thuộc mật độ CDM3Y4, 5, 6 và 8. Hai kịch bản đơn giản của độ phân cực spin
$\Delta$ phụ thuộc vào mật độ baryon đã được xem xét; đồng thời, kết quả của chúng được so
sánh với các ràng buộc từ các nghiên cứu về cấu trúc hạt nhân và quan sát sóng hấp dẫn
trong thiên văn (sự kiện GW170817 và GW190425). Các kịch bản tương ứng với phân cực một
phần ($\Delta < 1.0$) cũng tỏ ra hợp lí, với các kết quả của năng lượng đối xứng và độ nén
của chất hạt nhân đối xứng ứng với các tương tác hầu hết đều nằm trong các khoảng giá trị
thực nghiệm. Phương trình trạng thái của vật chất hạt nhân phi đối đứng dựa trên phương
pháp trường trung bình đã được áp dụng để xây dựng vật chất cân bằng \textbeta của sao
neutron, cấu thành bởi tương tác mạnh giữa các baryon (proton và neutron), electron và muon.

Phương trình trạng thái vật chất sao neutron sau đó được sử dụng làm đầu vào cho các tính
toán về các tính chất vĩ mô của sao neutron như khối lượng $M$, bán kính $R$, thông số
Love loại gravito-electric $k_\ell$ và gravito-magnetic $j_\ell$ ứng với các tương tác hạt nhân
kể trên cũng được tính bằng lý thuyết Tương đối rộng. Từ các ràng buộc thực nghiệm từ sự kiện GW170817 và giới hạn khối lượng của 2 pulsar nặng nhất từng được ghi nhận, ta đưa ra kết luận rằng chỉ các tương tác CDM3Y6 và CDM3Y8 là phù hợp nhất trong nghiên cứu về sao neutron. Các thông số Love gắn liền với biến dạng thủy triều của sao neutron trong các hệ sao đôi quay quanh nhau cũng được tính lên tới các đóng góp ở bậc 4 và mối liên hệ giữa giá trị của chúng với khối lượng sao cũng được chỉ ra.\\[5mm]
\textbf{Từ khóa:} Sao neutron, Magnetar, Phân cực spin, Phương trình trạng thái, Vật chất hạt nhân, Biến dạng thủy triều, Thông số Love, Sóng hấp dẫn.
