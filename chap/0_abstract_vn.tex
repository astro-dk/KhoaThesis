Phương trình trạng thái của vật chất sao neutron có phân cực spin trong từ trường mạnh đã được nghiên cứu bằng phương pháp trường trung bình trong mẫu Hartree-Fock bằng 6 tương tác hiệu dụng phụ thuộc mật độ CDM3Y3, 4, 5, 6, 8 và BDM3Y1. Một số kịch bản của độ phân cực spin phụ thuộc mật độ đã được xem xét (kịch bản A và B với các độ phân cực khác nhau ở bề mặt sao), đồng thời, kết quả của chúng được so sánh với các ràng buộc từ quan sát sóng hấp dẫn trong thiên văn (sự kiện GW170817 và GW190425) và cấu trúc hạt nhân. Các kết quả của năng lượng đối xứng $S$ và nặng lượng liên kết riêng $E/A$ của chất hạt nhân đối xứng ứng với các tương tác hầu hết đều nằm trong ràng buộc GW170817, trừ phiên bản CDM3Y3 và CDM3Y4. Các trường hợp phân cực spin một phần của kịch bản A đồng thời cũng tỏ ra phù hợp hơn khi so sánh các giá trị tính được của hệ số đối xứng $J$, tham số độ dốc $L$, độ cong $K_{sym}$ và độ nén $K$ với trong những trường hợp này đều gần hơn rất nhiều với khoảng giá trị được ràng buộc.\par
Ngoài ra, các tính chất vĩ mô của sao neutron như khối lượng $M$, bán kính $R$, thông số Love loại gravito-electric $k_l$ và gravito-magnetic $j_l$ ứng với các tương tác hạt nhân kể trên cũng được tính bằng lý thuyết Tương đối rộng. Các ràng buộc từ sự kiện GW170817 và giới hạn khối lượng của 2 pulsar nặng nhất từng được ghi nhận cũng đã chỉ ra rằng chỉ các tương tác CDM3Y6, CDM3Y8 và BDM3Y1 trong các trường hợp kể trên mới thỏa mãn được các kết quả thực nghiệm này. Chi tiết hơn, các thông số Love gắn liền với biến dạng thủy triều của sao neutron trong các hệ sao đôi quay quanh nhau cũng được tính lên tới các đóng góp ở bậc 4 và giá trị của chúng với khối lượng sao cũng được thảo luận chi tiết.\\[5mm]
\textbf{Từ khóa:} Sao neutron, Magnetar, Phân cực spin, Phương trình trạng thái, Vật chất hạt nhân, Biến dạng thủy triều, Thông số Love, Sóng hấp dẫn.
